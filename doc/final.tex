\documentclass[UTF8]{ctexart}
\usepackage{listings}
\usepackage{booktabs}  
\usepackage{geometry}  
\usepackage{graphicx} 
\usepackage{xcolor}
\usepackage{float}
\usepackage{array}
\usepackage{enumitem}
\usepackage{amsmath,amssymb,bm}
\usepackage[colorlinks=true]{hyperref}
\usepackage[version=4]{mhchem}
\usepackage{siunitx}
\usepackage{tikz}
\usepackage{pgfplots}
\pgfplotsset{compat=1.18}
\graphicspath{{figure/}} % 指定放置图片的子文件夹路径
\geometry{a4paper, left=2.5cm, right=2.5cm, top=2.5cm, bottom=2.5cm}
\definecolor{codegreen}{rgb}{0,0.6,0}
\definecolor{codegray}{rgb}{0.5,0.5,0.5}
\definecolor{codepurple}{rgb}{0.58,0,0.82}
\lstset{
    basicstyle=\ttfamily\footnotesize,
    breaklines=true,
    frame=single,
    numbers=left,
    numberstyle=\tiny\color{codegray},
    keywordstyle=\color{blue},
    commentstyle=\color{codegreen},
    stringstyle=\color{codepurple},
    showstringspaces=false
}


\begin{document}
\title{计算流体力学期末大作业}
\author{朱林-2200011028}
\date{\today}
\maketitle

\section{数理算法原理}
\subsection{问题描述}
\subsubsection{物理情形}
Sod激波管问题是一个一维理想气体流动问题:无限长管道中,初始时刻($t=0$)在$x=0$处有一薄膜分隔两侧气体:
\begin{itemize}
    \item 左侧($x<0$): 高压区,状态为$(\rho_L, u_L, p_L)$
    \item 右侧($x>0$): 低压区,状态为$(\rho_R, u_R, p_R)$
\end{itemize}

薄膜在$t=0^+$时刻瞬时破裂,两侧气体开始相互作用,产生复杂的波系结构。

\subsubsection{标准初始条件}
采用以下无量纲初始条件:
\begin{align*}
\text{左侧:} & \quad \rho_L = 1.0,  u_L = 0.0,  p_L = 1.0 \\
\text{右侧:} & \quad \rho_R = 0.125,  u_R = 0.0,  p_R = 0.1
\end{align*}

\subsection{控制方程}
流动由一维欧拉方程描述:
\begin{align}
&\frac{\partial \mathbf{U}}{\partial t} + \frac{\partial f(\mathbf{U})}{\partial x} = 0 \\
&\mathbf{U} = \begin{bmatrix} \rho \\ \rho u \\ E \end{bmatrix}, \quad
f(\mathbf{U}) = \begin{bmatrix} \rho u \\ \rho u^2 + p \\ u(E + p) \end{bmatrix}
\end{align}
其中总能密度$E = \rho e = \rho (C_v T + \frac{1}{2}u^2)$。

\subsection{Riemann问题精确解}
通过查阅资料,Sod激波管问题的精确解可以通过黎曼问题的解法得到。该问题的解由四个区域组成,分别对应不同的流动状态和波系结构。
\subsubsection{波系结构}



\subsubsection{解析解表达式}



作为黎曼问题的最简单形式,Sod激波管是理解更复杂流动机理的基础。同时常用于计算流体中作为典型案例检验算法和格式。



%附录
\newpage
\appendix
\section{AI工具使用声明表}
\begin{table}[H]
    \centering
    \begin{tabular}{c|c|c}
        \hline
        使用内容 & 工具名称 & 使用目的 \\ \hline
        hw2.tex 1-9行、图片插入 & Github Copilot & 调整pdf格式,调用宏包,省略插入图片的重复性工作 \\ 
        main.py 6-15行 & DeepSeek & 修正 matplotlib 中文显示问题 \\ 
        ReadMe.md框架 & DeepSeek & 在DeepSeek的帮助下生成一个框架,在此基础上增加而来 \\
        .gitignore & Github Copilot & 针对于python和latex的.gitignore文件,完全由Copilot生成  
    \end{tabular}
    \label{tab:AI_tools}
\end{table}
\end{document}